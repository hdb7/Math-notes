%% ----------------------------------------------------
%%              NOTES-TEMPLATE
%% ----------------------------------------------------
\documentclass[12pt,a4paper]{article}
\usepackage[margin=1in]{geometry}
\usepackage{graphicx}
\graphicspath{ {./images/} }
\usepackage{amsmath}
\usepackage{amsthm} %for proof
\usepackage{amssymb}
\usepackage{gensymb}
\usepackage{mathrsfs}
\usepackage{multicol}
\usepackage{tikz}
\usetikzlibrary{positioning}
\usepackage[autostyle]{csquotes}
\renewcommand{\baselinestretch}{0.8}
\setlength{\columnsep}{1cm}
\usepackage{xcolor} % font coloring
\usepackage{sectsty}
%\sectionfont{\color{red}}  % sets colour of sections
\setlength{\parindent}{-1em}
%\sectionfont{\color{red}}  % sets colour of sections
%\subsectionfont{\color{purple}}
%\subsubsectionfont{\color{blue}}

%% math macros
\newcommand{\R}{\mathbb{R}}
\newcommand{\Z}{\mathbb{Z}}
\newcommand{\N}{\mathbb{N}}
\newcommand{\C}{\mathbb{C}}
\newcommand{\Q}{\mathbb{Q}}
\newcommand{\W}{\mathbb{W}}
\newtheorem{thm}{Theorem}
\newtheorem{defn}{Definition}
\newtheorem{conv}{Convention}
\newtheorem{rem}{Remark}
\newtheorem{lem}{Lemma}
\newtheorem{cor}{Corollary}
\newtheorem{ex}{Example}

%%% Coloring macros
\newcommand{\cred}[1]{\textcolor{red}{#1}}
\newcommand{\cblue}[1]{\textcolor{blue}{#1}}
\newcommand{\ccyan}[1]{\textcolor{cyan}{#1}}
\newcommand{\cgreen}[1]{\textcolor{green}{#1}}
\newcommand{\cyellow}[1]{\textcolor{yellow}{#1}}
\newcommand{\cpurple}[1]{\textcolor{purple}{#1}}
\newcommand{\corange}[1]{\textcolor{orange}{#1}}
%% custom colors
\definecolor{astral}{RGB}{46,116,181}
\definecolor{darkblue}{RGB}{50,76,168}
\definecolor{darkbrown}{RGB}{99,12,8}

\title{Elementary Number Theory \vspace{0.5em}}
\author{Hamjak Debbarma}
\date{\today}
\linespread{0.5}

\begin{document}
  \maketitle
  \section*{Notations}
  $\Z = \{... -3,-2,-1,0,1,2,3...\}$ \\
  $\Z^{+} = \{1,2,3...\}$ \\
  $\Z^{-} = \{... -3,-2,-1\}$ \\
  $\Z^{0+} = \{0,1,2,3...\}$ \\
  $\Z^{0-} = \{...-2,-1,0\}$
  \section{Introduction: Divisibility, Prime, GCD}
  Let $a,b \in \Z$ and $a>0$ we say $a$ divides $b$, $a|b$ if $b=ac$ for some $c\in \Z$. Here, $a$ is the divisor or factor of $b$ and $b$ is the multiple of $a$.
  \\ \\
 \textbf{Note: }
  \begin{itemize}
  	\item Sign has no effects: $6|12, -5|53, 9|-81$
  	\item Divisibility is a statement not an operator like divide $/$
  	\item Divisibility is mostly deals with Positive Integers.
  \end{itemize}
\textbf{Properties of Divisibility}\\
Let $a,b,c \in \Z$, then
\begin{itemize}
	\item If $a|b$ and $a|c$ then $a|b+c$
	\item If $a|b$ and $b|c$ then $a|c$
	\item If $a|b$ then $a|mb$ for some integer $m$
	\item If $a|b$ and $a|c$ then $a|bm+cn$ for some integer $m,n$ 
\end{itemize}

\begin{defn}[Prime]
	Let $p>0$ and $p\in \Z^{+}$, $p$ is prime iff the divisor of $p$ is $1$ and $p$.
\end{defn}

\begin{defn}[Composite]
	Let $M>1$ which is not prime is composite.
\end{defn}
\begin{rem}
	$0$ and $1$ are neither prime nor composite.
\end{rem}
\begin{thm}[Fundamental Theorem of Arithmetic]
	Any integer greater than $1$ can be written as a unique product of primes. Here, the primes ordering does not matter.
\end{thm}

\begin{defn}[Common Divisor]
	The integer $c$ is the common divisor of $a$ and $b$ if $a=cn$ and $b=cm$ for some integer $n,m$ or if $c|a$ and $c|b$.
\end{defn}
\begin{defn}[GCD]
	gcd(a,b) is the largest common divisor of $a$ and $b$, $gcd(a,b)>1$ and by convention $a,b \ne 0$.
\end{defn}
\begin{defn}[Co-primes]
	If $gcd(a,b)=1$ then $a,b$ are relatively prime or coprime though $a,b$ needs not be prime.
\end{defn}

\begin{lem}[B\'ezout's Identity]
	If $gcd(a,b)=d$ then $ \exists x,y \in \Z$ s.t $ax+by=d$
\end{lem}
\begin{lem}
	if $a=bq+r$ then $gcd(a,b)=gcd(b,r)$
\end{lem}
\begin{lem}
	if $a|c$ and $b|c$ and $gcd(a,b)=1$ then $ab|c$
\end{lem}
\begin{thm}[Division]
	Let $a,b \in \Z$ , $b>0$ then $\exists q,r \in \Z$ s.t $a=bq+r$ where, $0\le r < b$
\end{thm}
 
\begin{defn}[Linear Diophantine Eqn]
	Given $a,b,c \in \N $ the eqn. $ax+by=c$ has a solution for $x,y \in \Z$ iff $gcd(a,b)|c$
\end{defn} 
 Note: To solve Diophantine we can used Extended Euclid's Algorithm.
\section{Congruences}
\begin{defn}
	Let $n$ be fixed positive integer, $a,b \in Z$ are said to be congruent modulo $n$, $a \equiv b \pmod n$ if $n|(a-b)$ i.e, $a-b = nk$ for some $k \in \Z$.
\end{defn}
\begin{ex}
	$n=7$, $3 \equiv 24 \pmod 7$ $\implies 7|(3-24) \implies 7|-21$
\end{ex}
\begin{ex}
 $6 \not \equiv 1 \pmod 3$ $\implies 3 \not |(6-1) $
\end{ex}

\textbf{Note}
\begin{itemize}
	\item Any two integers are congruent modulo $1$, $a\equiv b \pmod 1 \impliedby 1|(a-b)$
	\item Two integers are congruent modulo $2$ if either both even or both odd.
\end{itemize}

\begin{defn}[Equivalence Class]
	For $x \in \Z$ define the equivalence class of $x$ w.r.t $\equiv \pmod n$ by $[x]=\{a\in \Z | a\equiv x \pmod n \}$
\end{defn}
\textbf{Fact:} There are exactly $n$ equivalence classes modulo $n$ i.e, $[0],[2],\dots [n-1]$ that is, every integer is in one of those classes.
\begin{lem}	
	If $n>1$ and $a$ be any integers and $r$ be remainder when $a/n$ then $a \equiv r \pmod n$ or $\forall a \in \Z$ $a$ is congruent to exactly one of those least residue modulo $n$.
\end{lem}
\begin{proof}
	$a/n \implies a=qn+r$ where, $ q,r \in \Z $ and $0\le r < n$ \\
	$a-r = qn \implies a \equiv r \pmod n$
\end{proof}

\begin{cor}
	If $a \equiv r \pmod n$ then $r = \{0,1,2,\dots n-1\}$
\end{cor}

\begin{defn}[Complete System of Residue (CSR)]
	Given $a \in \Z$ let $q$ and $r$ be its quotient and remainder upon division by $n$ i.e, $a = qn+r, 0 \ge r < n$. Then by definition of congruences $a \equiv r \pmod n$ and $r = \{0,1,2,\dots n-1\}$ called the least non negative residue(remainder) modulo $n$. \\
	In general a collection of $\{a_{1}, a_{2}, \dots, a_{n}\}$ is a \textbf{Complete System of Residue} modulo $n$ if each $a_{i} \equiv r_{i} \pmod n$ i.e, $\{a_{1}, a_{2}, \dots, a_{n}\} \equiv \{0,1,2 \dots n-1\} \pmod n$ and $a_{i} \not \equiv a_{j} \pmod n$
\end{defn}

\begin{ex}
	Consider $n=4$ and $S = \{12,11,8,3\}$ does S form CSR modulo $4$.
\end{ex}
\begin{proof}[Soln.]
	$r = \{0,1,2,3\}$ and $12 \equiv 0 \pmod 4$ and $8 \equiv 0 \pmod 4$ implies, $12 \equiv 8 \pmod 4$. So, S does not form CSR.
\end{proof}

\begin{thm}
	For arbitrary integers $a$ and $b$, $a \equiv n \pmod n$ iff $a$ and $b$ leaves the same non-negative remainder when divided by $n$. 
\end{thm}
\begin{proof}
	$ a \equiv b \pmod n \implies a - b = nk \implies a = b + nk$ for some $k \in \Z$\\
	$n|b \implies b = nq + r \implies a = nq+r+nk \implies a = (nq+nk)+r$\\
	Now, assume $a=nq_{1}+r$ and $b=nq_{2}+r$ then $a-b=nq_{1}+r - nq_{2}-r \implies a-b = n(q_{1}-q_{2}) \implies a\equiv b \pmod n$
\end{proof}

\begin{thm}
	Let $n>1$ and $a,b,c,d \in \Z$ then the following properties hold :
	\begin{enumerate}
		\item $a \equiv a \pmod n$
		\item if  $a \equiv b \pmod n$ then  $b \equiv a \pmod n$
		\item if  $a \equiv b \pmod n$ and  $b \equiv c \pmod n$ then  $a \equiv c \pmod n$
		\item if  $a \equiv b \pmod n$ and  $c \equiv d \pmod n$ then  $a+c \equiv b+d \pmod n$ and  $ac \equiv bd \pmod n$
		\item  if $a \equiv b \pmod n$ then  $a+c \equiv b+c \pmod n$ and  $ac \equiv bc \pmod n$
		\item if  $a \equiv b \pmod n$ then  $a^{k} \equiv b^{k} \pmod n$ for any $k \in \Z^{+}$
	\end{enumerate}	
\end{thm}

\begin{proof}
	($Prop 5$) Using prop$1$ and prop$4$ $a \equiv b \pmod n$ and $ c \equiv c \pmod n$ implies, $a+c \equiv b+c \pmod n$ and $ac \equiv bc \pmod n$ \\
	($prop 6$) Using prop$4$ we can established the prove.
\end{proof}

\textbf{Divisibility Test for an Integer}
\begin{itemize}
	\item An integer is divisible by $2$ iff it's unit digits is $0,2,4,6,8$
	\item For by $3$ its digits sum should be divisible by $3$
	\item For $4$, the no. form by its last digits should be divisible by $4$
	\item For $5$, last digit should be $0$ or $5$.
	\item An integer $N$ is divisible by $6$ iff $6|M$, where $M = a_{0}+4a_{1}+\dots +4a_{m}$
	\item For $8$, the no. formed by last three digits should be divisible by $8$. 
	\item For by $9$ its digits sum should be divisible by $9$
	\item For $10$, the last digit should be $0$
	\item For $11$, $11|N$ iff the altering sum of its digit is divisible by $11$. E.g, $N=639162513 \implies 3-1+5-2+6-1+9-3+6=22$
\end{itemize}

\section{Linear Congruences}
\begin{defn}
	An eqn, of the form $ax\equiv b \pmod n$ is called the Linear congruences.
\end{defn}
\begin{lem}
	$ax \equiv b \pmod n$ has a solution iff $d|b$, where $d=gcd(a,n)$. If $d|b$ then it has $d$ mutually incongruent solution modulo $n$.
\end{lem}
\textbf{Note: }
\begin{enumerate}
	\item if $x_{1}$ is a soln of $ax \equiv b \pmod n$ then any other $x_{2} \equiv x_{1}  \pmod n$ is congruent solution.
	\item if $x_{1}$ and $x_{2}$ are both soln and  $x_{1} \not \equiv x_{2}  \pmod n$ then it is called incongruent soln of $ax \equiv b \pmod n$.
\end{enumerate}

\begin{defn}[Inverse of Modulo $n$]
Any value of $x$ which is a solution of $ax \equiv 1 \pmod n$ is called the inverse of modulo $n$. Thus if $a^{-1}$ is the inverse then $aa^{-1} \equiv 1 \pmod n$
\end{defn}

\textbf{Strategy for solving: $ax \equiv b \pmod n$}
\begin{enumerate}
	\item $a$ is invertible modulo $n$ iff $gcd(a,n)=1$, $ax+ny=1$ so, $ax\equiv 1 \pmod n$
	\item Reduction: if $ca\equiv cb \pmod n \implies a \equiv b \pmod {\frac{n}{gcd(c,n)}}$
	\item $ax \equiv b \pmod n$ has a soln iff $gcd(a,n)|b$
	\item if $ax \equiv b \pmod n$ has a soln then there are $gcd(a,n)$ number of soln separated by $\frac{n}{gcd(a,n)}$
\end{enumerate}

\section{Fermat's Little Theorem}
\section{Chinese Remainder Theorem}
\section{Wilson's Theorem}
\section{Hensel's Lemma}
\section{Polynomial Congruences}
\section{Langrage's Polynomial}
\section{Euler's Totient}
\end{document}
