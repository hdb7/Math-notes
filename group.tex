%% ----------------------------------------------------
%%              NOTES-TEMPLATE
%% ----------------------------------------------------
\documentclass[12pt,a4paper]{article}
\usepackage[margin=1in]{geometry}
\usepackage{graphicx}
\graphicspath{ {./images/} }
\usepackage{amsmath}
\usepackage{amssymb}
\usepackage{gensymb}
\usepackage{mathrsfs}
\usepackage{multicol}
\usepackage{tikz}
\usetikzlibrary{positioning}
\usepackage[autostyle]{csquotes}
\renewcommand{\baselinestretch}{0.8}
\setlength{\columnsep}{1cm}
\usepackage{xcolor} % font coloring
\usepackage{sectsty}
%\sectionfont{\color{red}}  % sets colour of sections
\setlength{\parindent}{-1em}
%\sectionfont{\color{red}}  % sets colour of sections
%\subsectionfont{\color{purple}}
%\subsubsectionfont{\color{blue}}

%% math macros
\newcommand{\R}{\mathbb{R}}
\newcommand{\Z}{\mathbb{Z}}
\newcommand{\N}{\mathbb{N}}
\newcommand{\C}{\mathbb{C}}
\newcommand{\Q}{\mathbb{Q}}
\newcommand{\W}{\mathbb{W}}
\newtheorem{thm}{Theorem}
\newtheorem{defn}{Definition}
\newtheorem{conv}{Convention}
\newtheorem{rem}{Remark}
\newtheorem{lem}{Lemma}
\newtheorem{cor}{Corollary}
\newtheorem{ex}{Example}

%%% Coloring macros
\newcommand{\cred}[1]{\textcolor{red}{#1}}
\newcommand{\cblue}[1]{\textcolor{blue}{#1}}
\newcommand{\ccyan}[1]{\textcolor{cyan}{#1}}
\newcommand{\cgreen}[1]{\textcolor{green}{#1}}
\newcommand{\cyellow}[1]{\textcolor{yellow}{#1}}
\newcommand{\cpurple}[1]{\textcolor{purple}{#1}}
\newcommand{\corange}[1]{\textcolor{orange}{#1}}
%% custom colors
\definecolor{astral}{RGB}{46,116,181}
\definecolor{darkblue}{RGB}{50,76,168}
\definecolor{darkbrown}{RGB}{99,12,8}

\title{Group Theory \vspace{-2em}}
%\author{Hamjak Debbarma}
\date{\today}
\linespread{0.5}

\begin{document}
  \maketitle
  
  \begin{defn}[Binary Operation]
  	$*: A \to A$
  \end{defn}
  
  \begin{defn}
  	A non-empty set $S$ w.r.t binary operation $*$ is called an algebraic structures. E.g, $S=\{1,-1\}$ is a structure under $\times$.
  \end{defn}
  
 \begin{defn}
 	    Let $(G,*)$ be an algebraic structures with binary operation $*$ then $G$ is called a group if it satisfy the following properties :
 	\begin{itemize}
 		\item Closure property i.e, $\forall a,b \in G\; a*b\in G$
 		\item There exist an identity element $e$ s.t $a*e=e*a=a\; \forall a\in G$
 		\item Every element $a\in G$ has an inverse $a^{-1}\in G$ s.t $a*a^{-1}=e$
 		\item $*$ is Associative i.e, $\forall a,b,c \in G\; (a*b)*c=a*(b*c)$
 	\end{itemize}
 \end{defn}
\begin{ex}
	$(M_{m\times n}(\R), +)$ is a group.
\end{ex}
\begin{ex}
	$G=\{M_{n\times n}(\R) | det(M) \ne 0 \}$ is a non-singular matrix, $(G,\times)$ is a group.
\end{ex}
\begin{defn}
	A group is $G$ is called Commutative or Abelian Group if it has commutative property i.e, $\forall a,b\in G\; a*b=b*a$
\end{defn}
\begin{defn}
	$(G,*)$ is a semigroup if it has - closure and Associativity.
\end{defn}
\begin{defn}
	$(G,*)$ is a monoid if it has - closure, Associativity and Identity element.
\end{defn}

\begin{defn}
	Let $(S,*)$ be a structure in which $S$ is non-empty set and $*$ is a binary operation define on $S$. Such structure is called groupoid.
\end{defn}

\begin{defn}
	A group $G$ is called a finite group if the no. of elements in $G$ is finite.
\end{defn}
\begin{defn}
	The no. of elements in a group (finite or infinite) is called is called its order, denoted as $o(G)$ or $|G|$
\end{defn}
\begin{ex}
	$G=\{1,-1,i,-i\} \implies o(G)=4$ 
\end{ex}
\begin{defn}
	The order of an element $a \in G$ is the smallest $n \in \Z^+$ s.t $a^n=e$, if $a^n \ne e$ then $a$ has infinite orders, denoted as $o(a)=n$ or $|a|=n$. For $+$ opereration $na=0.$
\end{defn}
\begin{ex}
	$G=\{1,-1,i,-i\}$ for $i \in G \implies o(i)=4 \impliedby i^4=(i^2)^2=(-1)^2=1$
\end{ex}

\begin{thm}[Cancellation Law]
	  Let $G$ be a group and $a,b,c\in G$ such that $a*b=a*c \implies b=c$
\end{thm}
\begin{cor}
	A group $G$ has a unique identity element.
\end{cor}
\begin{cor}
	Any element of $G$ has a unique inverse.
\end{cor}
\begin{thm} 
	If $a\in G$ then $(a^{-1})^{-1}=a$ and if $a,b\in G$ then $(ab)^{-1}=b^{-1}a^{-1}$
\end{thm}

\begin{defn}[Subgroup]
	Let $(G,*)$ be a group and if $H \subseteq G$ then $H$ is a subgroup of $G$ if $(H,*)$ is a group.
\end{defn}
\begin{ex}
	$2\Z$ is subgroup of $(\Z, +) \implies n\Z \subseteq \Z$
\end{ex}
\begin{defn}
	$g\in G$ is a generator of a group $G$ if $G=\{g^n | n\in \Z\}$
\end{defn}
\begin{ex}
	$1 \in \Z$ is a generator of $(\Z, +)$
\end{ex}

\begin{defn}
	A group $G$ is finitely generated if $\exists$ $g_{1},g_{2},\dots g_{k}\in G$ s.t every element of $G$ can be written as $g_{1}^{{\alpha}_{1}} \dots g_{k}^{{\alpha}_{k}} \in G  $ and $\alpha_{1}, \dots \alpha_{k} \in \Z$
\end{defn}
\begin{ex}
\end{ex}	
\begin{itemize}
	\item $(\Z, +)$
	\item $G = \{\frac{a}{b} | a,b \text{ consist of prime} \le r\}$ is finitely generated.
\end{itemize}

\end{document}
