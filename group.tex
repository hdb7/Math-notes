%% ----------------------------------------------------
%%              NOTES-TEMPLATE
%% ----------------------------------------------------
\documentclass[12pt,a4paper]{article}
\usepackage[margin=1in]{geometry}
\usepackage{graphicx}
\graphicspath{ {./images/} }
\usepackage{amsmath}
\usepackage{amssymb}
\usepackage{amsthm}
\usepackage{gensymb}
\usepackage{mathrsfs}
\usepackage{multicol}
\usepackage{tikz}
\usetikzlibrary{positioning}
\usepackage[autostyle]{csquotes}
\renewcommand{\baselinestretch}{0.8}
\setlength{\columnsep}{1cm}
\usepackage{xcolor} % font coloring
\usepackage{sectsty}
%\sectionfont{\color{red}}  % sets colour of sections
\setlength{\parindent}{-1em}
%\sectionfont{\color{red}}  % sets colour of sections
%\subsectionfont{\color{purple}}
%\subsubsectionfont{\color{blue}}

%% math macros
\newcommand{\R}{\mathbb{R}}
\newcommand{\Z}{\mathbb{Z}}
\newcommand{\N}{\mathbb{N}}
\newcommand{\C}{\mathbb{C}}
\newcommand{\Q}{\mathbb{Q}}
\newcommand{\W}{\mathbb{W}}
\newtheorem{thm}{Theorem}
\newtheorem*{defn}{Definition}
\newtheorem{conv}{Convention}
\newtheorem*{rem}{Remark}
\newtheorem*{lem}{Lemma}
\newtheorem*{cor}{Corollary}
\newtheorem{ex}{Example}
\newtheorem{prop}{Proposition}

%%% Coloring macros
\newcommand{\cred}[1]{\textcolor{red}{#1}}
\newcommand{\cblue}[1]{\textcolor{blue}{#1}}
\newcommand{\ccyan}[1]{\textcolor{cyan}{#1}}
\newcommand{\cgreen}[1]{\textcolor{green}{#1}}
\newcommand{\cyellow}[1]{\textcolor{yellow}{#1}}
\newcommand{\cpurple}[1]{\textcolor{purple}{#1}}
\newcommand{\corange}[1]{\textcolor{orange}{#1}}
%% custom colors
\definecolor{astral}{RGB}{46,116,181}
\definecolor{darkblue}{RGB}{50,76,168}
\definecolor{darkbrown}{RGB}{99,12,8}

\title{Group Theory \vspace{-2em}}
%\author{Hamjak Debbarma}
\date{\today}
\linespread{0.5}

\begin{document}
  \maketitle
  
  \section{Groups, Semigroups, Monoids and Groupoid}
  
  \begin{defn}[Binary Operation]
  	$*: A \to A$
  \end{defn}
  
  \begin{defn}
   A non-empty set $S$ w.r.t binary operation $*$ is called an algebraic structures. E.g, $S=\{1,-1\}$ is a structure under $\times$.
  \end{defn}
  
 \begin{defn}
 	    Let $(G,*)$ be an algebraic structures with binary operation $*$ then $G$ is called a group if it satisfy the following properties :
 	\begin{itemize}
 		\item Closure property i.e, $\forall a,b \in G\; a*b\in G$
 		\item There exist an identity element $e$ s.t $a*e=e*a=a\; \forall a\in G$
 		\item Every element $a\in G$ has an inverse $a^{-1}\in G$ s.t $a*a^{-1}=e$
 		\item $*$ is Associative i.e, $\forall a,b,c \in G\; (a*b)*c=a*(b*c)$
 	\end{itemize}
 \end{defn}
\begin{ex}
	$(M_{m\times n}(\R), +)$ is a group.
\end{ex}
\begin{ex}
	$G=\{M_{n\times n}(\R) | det(M) \ne 0 \}$ is a non-singular matrix, $(G,\times)$ is a group.
\end{ex}
\begin{defn}
	A group is $G$ is called Commutative or Abelian Group if it has commutative property i.e, $\forall a,b\in G\; a*b=b*a$
\end{defn}
\begin{defn}
	$(G,*)$ is a semigroup if it has - closure and Associativity.
\end{defn}
\begin{defn}
	$(G,*)$ is a monoid if it has - closure, Associativity and Identity element.
\end{defn}

\begin{defn}
	Let $(S,*)$ be a structure in which $S$ is non-empty set and $*$ is a binary operation define on $S$. Such structure is called groupoid.
\end{defn}

\begin{defn}
	A group $G$ is called a finite group if the no. of elements in $G$ is finite.
\end{defn}
\begin{defn}
	The no. of elements in a group (finite or infinite) is called is called its order, denoted as $o(G)$ or $|G|$
\end{defn}
\begin{ex}
	$G=\{1,-1,i,-i\} \implies o(G)=4$ 
\end{ex}
\begin{defn}
	The order of an element $a \in G$ is the smallest $n \in \Z^+$ s.t $a^n=e$, if $a^n \ne e$ then $a$ has infinite orders, denoted as $o(a)=n$ or $|a|=n$. For $+$ opereration $na=0.$
\end{defn}
\begin{ex}
	$G=\{1,-1,i,-i\}$ for $i \in G \implies o(i)=4 \impliedby i^4=(i^2)^2=(-1)^2=1$
\end{ex}

\begin{thm}[Cancellation Law]
	  Let $G$ be a group and $a,b,c\in G$ such that $a*b=a*c \implies b=c$
\end{thm}
\begin{cor}
	A group $G$ has a unique identity element.
\end{cor}
\begin{cor}
	Any element of $G$ has a unique inverse.
\end{cor}
\begin{thm} 
	If $a\in G$ then $(a^{-1})^{-1}=a$ and if $a,b\in G$ then $(ab)^{-1}=b^{-1}a^{-1}$
\end{thm}

%% ------------------------------------------------------------- %%

\section{Subgroups, Cosets and Normal Subgroup}

\begin{defn}[Subgroup]
	Let $(G,*)$ be a group and if $H \subseteq G$ then $H$ is a subgroup of $G$ if $(H,*)$ is a group. Denoted as $H \le G$
\end{defn}
\begin{defn}
	The trivial subgroup of any group is the subgroup $\{e\}$ consisting of just the identity element.
\end{defn}
\begin{defn}
	A proper subgroup of a group G is a subgroup H which is a proper subset of G i.e, $H \ne G$. Denoted as $H < G$
\end{defn}
\begin{defn}
	If H is a subgroup of G, then G is sometimes called an overgroup of H.
\end{defn}
\begin{ex}
	$2\Z$ is subgroup of $(\Z, +) \implies n\Z \le \Z$
\end{ex}

\begin{defn}
	Let $(G,*)$ be a group and $H<G$, and $g \in G$ then -
	\begin{itemize}
		\item The left coset of $H$ by $g$ is $gH = \{gh | h \in H\}$
		\item The right coset of $H$ by $g$ is $Hg = \{hg | h \in H\}$
	\end{itemize}
\end{defn}
\begin{lem}
	Each coset of a subgroup $H$ has the same size as $H$ i.e, $|gH|=|H|=|Hg|$
\end{lem}

\begin{rem}
Coset divides the finite group $G$ in equals parts that is $G = H \cup g_{1}H \cup \dots \cup g_{k}H$ . Therefore,
\begin{align*}
		|G| &= |eH| + |gH| + \dots + |g_{k}H| \\
			&= k|H|
\end{align*}
\end{rem}

\begin{defn}
	The no. of cosets (left or right) of $H$ in $G$ is called the index of $H$ in $G$. Denoted as $[G:H]$. Therefore, we have a counting formula as $$
		|G| = [G:H] \times |H|
	$$
\end{defn}
\begin{rem}
	The above serve as a proof of Lagrange's Theorem
\end{rem}
\begin{thm}[Lagrange's Theorem]
	Let $G$ be a finite group and $H \le G$ then $o(H)|o(G)$
\end{thm}
\begin{lem}
	The identity element and inverse of a subgroup is same as that of group.
\end{lem}

\begin{thm}[Two step subgroup test]
	Let $H \subseteq G$ , $H$ is a subgroup of $G$ iff -
	\begin{itemize}
		\item $a \in H, b\in H \implies a*b \in H$
		\item $a \in H \implies a*a^{-1} \in H$
	\end{itemize}
\end{thm}
\begin{defn}
	Let $H \le G$ then using $H$ we can form the cosets as $H, gH, g_{2}H \dots$ if all these cosets form a group when $g^{-1}Hg \in H$ for any $g\in G$ then $H$ is called the normal subgroup. Denoted as $H \trianglelefteq G$ .
\end{defn}
\begin{defn}
	The cosets group is called a Factor or Quotient Group. Denoted as $G/H$
\end{defn}
\begin{defn}
	If the only normal subgroups of $G$ are $G$ and $\{e\}$ then $G$ is called a simple group.
\end{defn}

\begin{thm}
	Every subgroup of Abelian group is a normal subgroup
\end{thm}
\begin{proof}
	Let $g\in G$ and $h \in H$ then
	\begin{align*}
		g^{-1}hg &= g^{-1}(gh)		&& \text{Since, $G$ is commutative} \\
			     &= (g^{-1}g)h \\
			     &= eh \implies h \in H
	\end{align*}
\end{proof}
\begin{cor}
	Every subgroup of a cyclic group is also a normal subgroup
\end{cor}
%% ------------------------------------------------------------ %%

\section{Cyclic Group}
\begin{defn}
	$g\in G$ is a generator of a group $G$ if $G=\{g^n | n\in \Z\}$
\end{defn}
\begin{ex}
	$1 \in \Z$ is a generator of $(\Z, +)$
\end{ex}

\begin{defn}
	A group $G$ is finitely generated if $\exists$ $g_{1},g_{2},\dots g_{k}\in G$ s.t every element of $G$ can be written as $g_{1}^{{\alpha}_{1}} \dots g_{k}^{{\alpha}_{k}} \in G  $ and $\alpha_{1}, \dots \alpha_{k} \in \Z$
\end{defn}
\begin{ex}
\end{ex}	
\begin{itemize}
	\item $(\Z, +)$
	\item $G = \{\frac{a}{b} | a,b \text{ consist of prime} \le r\}$ is finitely generated.
\end{itemize}


%%--------------------------------------------------------------------%%

\section{Commutative Group}
\begin{defn}[Partition of Integers]
	A multiset of positive integers that add to $n$ is called a partition of $n$. The no.of partitions of $k$ is denoted as $p(k)$.
\end{defn}
\begin{ex}
	$p(3) = 3 \impliedby \{3,1+2,1+1+1\}$ ($1+2$ is same as $2+1$)
\end{ex}
\begin{thm}[Fundamental Theorem of Finite Abelian Group]
	The no. of Abelian groups of order $n$ is the product of no. of partitions of $n_{i}$, where $n_{i}$ is obtained from the prime factorization of $n$ 
	$$
		n = p_{1}^{{n}_{1}}. p_{2}^{{n}_{2}}\dots p_{k}^{{n}_{k}}
	$$
\end{thm}

\section{Homomorphism and Isomorphism}


\end{document}
