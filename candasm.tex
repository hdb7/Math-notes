%% ----------------------------------------------------
%%              NOTES-TEMPLATE
%% ----------------------------------------------------
\documentclass[12pt,a4paper]{article}
\usepackage[margin=1in]{geometry}
\usepackage{graphicx}
\graphicspath{ {./images/} }
\usepackage{amsmath}
\usepackage{amssymb}
\usepackage{gensymb}
\usepackage{mathrsfs}
\usepackage{multicol}
\usepackage{tikz}
\usetikzlibrary{arrows, arrows.meta}
\usetikzlibrary{positioning}
\usepackage[autostyle]{csquotes}
\renewcommand{\baselinestretch}{0.8}
\setlength{\columnsep}{1cm}
\usepackage{xcolor} % font coloring
\usepackage{sectsty}
%\sectionfont{\color{red}}  % sets colour of sections
\setlength{\parindent}{-1em}
%\sectionfont{\color{red}}  % sets colour of sections
%\subsectionfont{\color{purple}}
%\subsubsectionfont{\color{blue}}

%% math macros
\newcommand{\R}{\mathbb{R}}
\newcommand{\Z}{\mathbb{Z}}
\newcommand{\N}{\mathbb{N}}
\newcommand{\C}{\mathbb{C}}
\newcommand{\Q}{\mathbb{Q}}
\newcommand{\W}{\mathbb{W}}
\newtheorem{thm}{Theorem}
\newtheorem{defn}{Definition}
\newtheorem{conv}{Convention}
\newtheorem{rem}{Remark}
\newtheorem{lem}{Lemma}
\newtheorem{cor}{Corollary}
\newtheorem{ex}{Example}

%%% Coloring macros
\newcommand{\cred}[1]{\textcolor{red}{#1}}
\newcommand{\cblue}[1]{\textcolor{blue}{#1}}
\newcommand{\ccyan}[1]{\textcolor{cyan}{#1}}
\newcommand{\cgreen}[1]{\textcolor{green}{#1}}
\newcommand{\cyellow}[1]{\textcolor{yellow}{#1}}
\newcommand{\cpurple}[1]{\textcolor{purple}{#1}}
\newcommand{\corange}[1]{\textcolor{orange}{#1}}
%% custom colors
\definecolor{astral}{RGB}{46,116,181}
\definecolor{darkblue}{RGB}{50,76,168}
\definecolor{darkbrown}{RGB}{99,12,8}

\title{C Programming and Assembly \vspace{-2em}}
%\author{Hamjak Debbarma}
\date{\today}
\linespread{0.5}

\begin{document}
  \maketitle
  
  \section{Basic overview of $\mu$P}
  The main purpose of $\mu$P is to execute a program store in a memory.\\
 \begin{figure}[h]
 	\includegraphics[width=8cm]{mp.png}
 	\centering
 \end{figure}
\\
\textbf{Address Bus:} logically address $2^N$ addresses where $N$ is the no. of address bits. \\
\textbf{Control Signal:} RD(Read), WR(Write)

\subsection{Logical Memory Map}
 Consider $ADDR[0:N-1]$, we have $2^N$ reference location and $DATA[0:k-1]$, $k$ bits of information in each locations. e,g $0,1, \dots 2^N-1$ location and each with $k=8$ bits of data

\subsection{Instruction Cycle}
\textbf{1. Fetch(F)}\\
\textbf{2. Decode(D)}\\
\textbf{3. Execute(E)} \\
Following cycle: $FI_1DI_1EI_1, FI_2DI_2EI_2 \dots FI_mDI_mEI_m$, where $I$ stands for Instructions \\
\textbf{For performing FDE what do we need ?} \\
For \textbf{FETCH: } fetching instructions from memory
\begin{itemize}
	\item Need something to store(Instructions)
	\item Instructions Pointer: \textit{always points to next instructions in the memory}
\end{itemize}
For \textbf{EXECUTE} we have following registers:


 
\end{document}
