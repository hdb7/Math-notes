%% ----------------------------------------------------
%%              NOTES-TEMPLATE
%% ----------------------------------------------------
\documentclass[12pt,a4paper]{article}
\usepackage[margin=1in]{geometry}
\usepackage{graphicx}
\graphicspath{ {./images/} }
\usepackage{amsmath}
\usepackage{amssymb}
\usepackage{gensymb}
\usepackage{mathrsfs}
\usepackage{multicol}
\usepackage{tikz}
\usetikzlibrary{arrows, arrows.meta}
\usetikzlibrary{positioning}
\usepackage[autostyle]{csquotes}
\renewcommand{\baselinestretch}{0.8}
\setlength{\columnsep}{1cm}
\usepackage{xcolor} % font coloring
\usepackage{listings}
\usepackage{sectsty}
%\sectionfont{\color{red}}  % sets colour of sections
\setlength{\parindent}{-1em}
%\sectionfont{\color{red}}  % sets colour of sections
%\subsectionfont{\color{purple}}
%\subsubsectionfont{\color{blue}}

%% math macros
\newcommand{\R}{\mathbb{R}}
\newcommand{\Z}{\mathbb{Z}}
\newcommand{\N}{\mathbb{N}}
\newcommand{\C}{\mathbb{C}}
\newcommand{\Q}{\mathbb{Q}}
\newcommand{\W}{\mathbb{W}}
\newtheorem{thm}{Theorem}
\newtheorem{defn}{Definition}
\newtheorem{conv}{Convention}
\newtheorem{rem}{Remark}
\newtheorem{lem}{Lemma}
\newtheorem{cor}{Corollary}
\newtheorem{ex}{Example}

%%% Coloring macros
\newcommand{\cred}[1]{\textcolor{red}{#1}}
\newcommand{\cblue}[1]{\textcolor{blue}{#1}}
\newcommand{\ccyan}[1]{\textcolor{cyan}{#1}}
\newcommand{\cgreen}[1]{\textcolor{green}{#1}}
\newcommand{\cyellow}[1]{\textcolor{yellow}{#1}}
\newcommand{\cpurple}[1]{\textcolor{purple}{#1}}
\newcommand{\corange}[1]{\textcolor{orange}{#1}}

%% ---Syntax color for Programming language ------
\definecolor{dkgreen}{rgb}{0,0.6,0}
\definecolor{gray}{rgb}{0.5,0.5,0.5}
\definecolor{mauve}{rgb}{0.58,0,0.82}

\lstset{frame=tb,
	language=C, %% -language
	aboveskip=3mm,
	belowskip=3mm,
	showstringspaces=false,
	columns=flexible,
	basicstyle={\small\ttfamily},
	numbers=none,
	numberstyle=\tiny\color{gray},
	keywordstyle=\color{blue},
	commentstyle=\color{dkgreen},
	stringstyle=\color{mauve},
	breaklines=true,
	breakatwhitespace=true,
	tabsize=4
}



\title{C Programming and Assembly \vspace{-2em}}
%\author{Hamjak Debbarma}
\date{\today}
\linespread{0.5}

\begin{document}
  \maketitle
  
  \section{Basic overview of $\mu$P}
  The main purpose of $\mu$P is to execute a program store in a memory.\\
 \begin{figure}[h]
 	\includegraphics[width=8cm]{mp.png}
 	\centering
 \end{figure}
\\
\textbf{Address Bus:} logically address $2^N$ addresses where $N$ is the no. of address bits. \\
\textbf{Control Signal:} RD(Read), WR(Write)

\subsection{Logical Memory Map}
 Consider $ADDR[0:N-1]$, we have $2^N$ reference location and $DATA[0:k-1]$, $k$ bits of information in each locations. e,g $0,1, \dots 2^N-1$ location and each with $k=8$ bits of data

\subsection{Instruction Cycle}
\textbf{1. Fetch(F)}\\
\textbf{2. Decode(D)}\\
\textbf{3. Execute(E)} \\
Following cycle: $FI_1DI_1EI_1, FI_2DI_2EI_2 \dots FI_mDI_mEI_m$, where $I$ stands for Instructions \\
\textbf{For performing FDE what do we need ?} \\
For \textbf{FETCH: } fetching instructions from memory
\begin{itemize}
	\item Need something to store(Instructions)
	\item Instructions Pointer: \textit{always points to next instructions in the memory}
\end{itemize}
For \textbf{EXECUTE} we have following general purpose registers: \\
$a) \frac{E| \frac{AH(8bit)|AL(8bit)}{\small{AX(16bit)}}}{EAX(32bit)}$
$b) \frac{E| \frac{BH(8bit)|BL(8bit)}{\small{BX(16bit)}}}{EBX(32bit)}$
$c) \frac{E| \frac{CH(8bit)|CL(8bit)}{\small{CX(16bit)}}}{ECX(32bit)}$
$d) \frac{E| \frac{DH(8bit)|DL(8bit)}{\small{DX(16bit)}}}{EDX(32bit)}$
\\
For array/string manipulation at the Hardware level:\\
1. Source Index(SI) Register: $\frac{E|SI(16bit)}{ESI(32bit)}$\\
2. Destination Index(DI) Register: $\frac{E|DI(16bit)}{EDI(32bit)}$
\\
\subsection{Partitioning of Memory}
Segmentation of Code, Data, Function in memory\\
MEMORY \fbox{
\fbox{CODE}
\fbox{DATA}
\fbox{STACK}
\fbox{EXTRA SEGMENT}
}
\\
Segment register are associated with: 
\begin{itemize}
	\item CODE: $E|CS$
	\item DATA: $E|DS$
	\item STACK: $E|SS$
	\item EXTRA SEGMENT: $E|ES$
\end{itemize}

\subsection{x$86$ Instructions}
\subsubsection{Data Transfer}
$1.\; MOV [DEST][SRC]$\\
Register direct addressing: $MOV AX, BX \;(AX \leftarrow BX)$ \\
Immediate addressing: $MOV AX, 0\times40 \;(AX \leftarrow 0\times40)$ \\
Direct addressing: $MOV\; 0\times1320, AX \;(0\times1320 \leftarrow AX)$ \\
Register indirect addressing: $MOV AX, [BX] \;(AX \leftarrow [BX])$ content of BX is moved to AX \\
\textbf{WORD-16 bit \& DWORD-32 bit}
\subsubsection{ALU Instruction}
$ADD \;AX, BX\; (AX \leftarrow AX+BX)$ \\
$SUB \;AX, BX\; (AX \leftarrow AX-BX)$ \\
$MUL \;[REG]$ : for multiplication it takes only one register the other is assume to be the prior register.\\
\begin{lstlisting}[language=Ant, caption=MUL example]
MOV AX, 0x4500	;AX <- 0x4500
MUL BX			; AX <- AX*BX
\end{lstlisting}
$INC \; AL|AX|EAX\; (AL\leftarrow AL+1)$\\
$DEC \; AL|AX|EAX\; (AL\leftarrow AL-1)$ \\
$CMP \; AX, BX \; (AX-BX)$ the value of this is evaluated using FLAG Register\\
\textbf{Clearing Register}
$XOR \; AX, BX \; (AX \leftarrow 0)$

\subsubsection{Stack Operation}
STACK POINTER (ESP) - Last In First Out (LIFO) Operation\\
$1.\; PUSH \;AL|AX|EAX\; [ESP \leftarrow ESP -1/2/4]$ \\ 
$2.\; POP\; AL|AX|EAX\; [ESP \leftarrow ESP + 1/2/4]$  \\ 
BASE POINTER (EBP) - Random access operation on stack\\

\newpage
\subsection{Call and Return Instructions}
\subsubsection{Subroutine (Function in C)}
CALL: two way branching goes to subroutine then comes back to where it left.\\
$EIP \leftarrow EIP+N$ where, $N$ is the address of next instructions to be executed.
\begin{lstlisting}[language=Ant, caption=subroutine example]
LOC_FUN:	ADD EBX, 0x0002;
			ADD EAX, 0x0003;
			SUB ECX, 0x0004;
			RET
MAIN:		XOR EBX, EBX;
			XOR EAX, EAX;
			XOR ECX, ECX;
			CALL LOC_FUN; 
			XOR EBX, EBX;
			XOR EAX, EAX;
			XOR ECX, ECX;
			CALL LOC_FUN; 		
\end{lstlisting}
When CALL LOC\textunderscore FUN is executed PUSH EIP is performed on stack and EIP $\leftarrow$ LOC\textunderscore FUN, RET simply pops the top of stack into EIP.

\subsection{Inline Assembly}
Interoperation between C ans Assembly. Consider the following C code
\begin{lstlisting}[language=C, caption=Inline Example]
	void main() {
		int x=2;
		x = x + 2;
		printf("%d", x);
	}
\end{lstlisting}
Say, we want to speed up $x = x + 2$ then we write the assembly version of those segment.
\begin{lstlisting}[language=C, caption=Inline Example]
	void main() {
		int x=2;
		//x = x + 2;
		__asm{
			MOV EAX, x		; EAX <- x
			ADD EAX, 0x002	; EAX <- EAX+2
			MOV x, EAX		; x <- EAX
		}
		printf("%d", x);	// 4
	}
\end{lstlisting}

\subsubsection{Commonly used Data Types in C}
\begin{itemize}
	\item BYTE OF DATA: char
	\item WORD OF DATA: short int or int
	\item DWORD OF DATA: long int or int
\end{itemize}

\newpage

\begin{lstlisting}[language=C, caption= repeatative addition]

main() {
	// z=x*y
	short int x=2, y=3;
	int z=0;
	__asm{
		XOR EAX, EAX // EAX <- 0
		MOV ECX, y
MULT:	ADD EAX, x
		DEC ECX		// ECX <- ECX-1
		JNZ MULT // as long as ECX != 0 repeat adding
		MOV z, EAX // z=x*y
		
	}
}
\end{lstlisting}

\begin{lstlisting}[language=C, caption= Optimised vs unoptimised]
int a,b,c,d;
int x=10,y=5;
a=x+y;
b=a-y;
c=b*y;
d=c/y
//---------- Un Optimised ---------//
mov x, 0x000a
mov y, 0x0005

mov eax,x
add eax,y
mov a,eax

mov ebx,a
sub ebx,y
mov b,ebx

mov eax,b
imul y
mov c,eax

mov ebx,c
idiv y
mov d,ebx

//------------- Optimised-------//
mov eax,x
add eax,y	// eax <- x+y
mov a,eax
sub ebx,y	// eax <- a-y
mov b,ebx
imul y		// eax <- b*y
mov c,eax
idiv y		// eax <- c/y
mov d,ebx
\end{lstlisting}

\newpage
\subsection{Pointer}

\begin{lstlisting}[language=C, caption= pointer arithmetic]

main() {
	char *pa=0; // mov pa,0x0000
	char *pb=0;	// mov pb,0x0000
	/*
	mov eax,pa
	add eax,0x0001
	mov pa,eax
	*/
	pa++;		// pa=pa+1
	/*
	mov eax,pb
	add eax,0x0004
	mov pb,eax
	*/
	pb++;		// pb=pb+4
	
}

\end{lstlisting}

\begin{lstlisting}[language=C, caption= character pointer]
main () {
	char *s = "This is all nonsense";
	int i=0;
	for(i;s[i]!='\0';i++){}
}
//-------------asm optimised----------//
int i=0;
int cnt=0;
__asm {
	mov ecx,0
	mov ebx,s
cm :cmp byte ptr [ebx],0x00
	jz done
	inc ecx
	in ebx
	jmp cm
done: mov cnt, ecx
}
\end{lstlisting}





\end{document}
